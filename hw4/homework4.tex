% LaTeX Article Template - customizing header and footer
\documentclass{article}

\newtheorem{thm}{Theorem}

% Set left margin - The default is 1 inch, so the following 
% command sets a 1.25-inch left margin.
\setlength{\oddsidemargin}{0.25in}

% Set width of the text - What is left will be the right margin.
% In this case, right margin is 8.5in - 1.25in - 6in = 1.25in.
\setlength{\textwidth}{6in}

% Set top margin - The default is 1 inch, so the following 
% command sets a 0.75-inch top margin.
\setlength{\topmargin}{-0.25in}

% Set height of the header
\setlength{\headheight}{0.3in}

% Set vertical distance between the header and the text
\setlength{\headsep}{0.2in}

% Set height of the text
\setlength{\textheight}{9in}

% Set vertical distance between the text and the
% bottom of footer
\setlength{\footskip}{0.1in}

% Set the beginning of a LaTeX document
\usepackage{multirow}

\usepackage{tikz}

\usepackage{fullpage}
\usepackage{graphicx}
\usepackage{amsthm}
\usepackage{url}
\usepackage{amssymb}
\usepackage{amssymb}
\usepackage{algpseudocode}
\graphicspath{%
    {converted_graphics/}% inserted by PCTeX
    {/}% inserted by PCTeX
}
%%%%%%%%%%%%%%%%%%%%%%%%%%%%%




\begin{document}\title{Homework $4$\\ Computer Science \\ B351 Spring 2017\\ Prof. M.M. Dalkilic}         % Enter your title between curly braces
\author{Jonathon Cooke-Akaiwa}        % Enter your name between curly braces
\date{\today}          % Enter your date or \today between curly braces
\maketitle


% Redefine "plain" pagestyle
\makeatother     % `@' is restored as a "non-letter" character




% Set to use the "plain" pagestyle
\pagestyle{plain}

All the work herein is mine.
\section*{Homework Questions}
\begin{enumerate}
\item Convert the following logical sentences to clausal form:
\begin{enumerate}
\item $\exists\ y\ p(y) \vee [\exists\ y\ (q(y) \rightarrow (\exists\ x\ (p(x) \vee\ q(x,y,C)))]$ \newline
$\exists\ y\ p(y) \vee [\exists\ y\ (\neg q(y) \vee (\exists\ x\ (p(x) \vee\ q(x,y,C)))]$ \newline
$\exists\ y\ p(y) \vee [\exists\ z\ (\neg q(z) \vee (\exists\ x\ (p(x) \vee\ q(x,z,C)))]$ \newline
$\exists\ y\ \exists\ z\ p(y) \vee (\neg q(z) \vee (\exists\ x\ (p(x) \vee\ q(x,z,C)))$ \newline
$\exists\ y\ \exists\ z\ \exists\ x\ p(y) \vee (\neg q(z) \vee ((p(x) \vee\ q(x,z,C)))$ \newline
$\exists\ z\ \exists\ x\ p(f()) \vee (\neg q(z) \vee ((p(x) \vee\ q(x,z,C)))$ \newline
$\exists\ x\ p(f()) \vee (\neg q(g()) \vee ((p(x) \vee\ q(x,g(),C)))$ \newline
$p(f()) \vee (\neg q(g()) \vee ((p(h()) \vee\ q(h(),g(),C)))$ \newline
$p(f()), (\neg q(g()) , ((p(h()) ,\ q(h(),g(),C)))$ \newline
$[p[f[]], [\neg, q,[g,[]]] , [p,[h,[]]] ,\ [q[h[],g[],C]]]$ \newline



\item $\forall x \forall y \forall x\ d(x,y) \wedge d(y,z) \rightarrow d(x,z)$
\item $( P \vee Q) \wedge (\neg P \rightarrow (Q \vee R))$ \newline
	$( P \vee Q) \wedge (\neg \neg P \vee (Q \vee R))$ \newline
	$( P \vee Q) \wedge (P \vee (Q \vee R))$ \newline
	$[\wedge, [\vee, P, Q], [\vee, P , [\vee, Q, R]]]$
	
\end{enumerate}
\item Let $\mathcal{U} = \{1,2,3\}$, $p = \{1,3\}$, $m = \{(1,1),(2,1),(3,2)\}$
\begin{enumerate}
\item Determine $\models \forall\ x\ \exists\ y\ m(x,y)$ \newline
for x = 1 \newline
(1,1) \newline
for x = 2
(2,1) \newline
for x = 3
(3,2) \newline
Therefore $\models \forall\ x\ \exists\ y\ m(x,y)$ is valid.
\item Determine $\models \forall\ y\ \exists\ x\ m(x,y)$ \newline
for y = 1 \newline
(1,1),(2,1) \newline
for y = 2 \newline
(3,2) \newline
for y = 3 \newline
\{\} \newline
therefore $\models \forall\ y\ \exists\ x\ m(x,y)$ is not valid.
\item Determine $\models \forall\ x\ \exists\ x\ m(x,x)$ \newline
for x = 1 \newline
(1,1) \newline
for x = 2 \newline
\{\} \newline
Therefore $\models \forall\ x\ \exists\ x\ m(x,x)$ is not valid.
\item Determine $\models \exists \ x\ \forall\ y\ m(x,y)$ \newline
for x = 1 \newline
(1,1) \newline
for x = 2 \newline
(2,1) \newline
for x = 3 \newline
(3,2) \newline
Because none of these x values holds all there y values, $\models \exists \ x\ \forall\ y\ m(x,y)$ is not valid.
\item Determine $\models \exists \ x\ \forall\ y\ m(y,x)$ \newline
y = 1 \newline
(1,1) \newline
y = 2 \newline 
(2,1) \newline 
y = 3 \newline 
(3,2) \newline
Therefore $\models \exists \ x\ \forall\ y\ m(y,x)$ is valid.
\item Determine $\models \exists \ x\ \forall\ x\ m(x,x)$ \newline
for x = 1 \newline
(1,1) \newline
Therefore $\models \exists \ x\ \forall\ x\ m(x,x)$ is valid.

\end{enumerate}
\item You've decided to add a new quantifier: $M$ that takes one variable.  The syntax is $M\ x\ f(x)$ for some sentence $f$.  The meaning of $M\ x\ f(x)$ is that the number of times $\sigma(u,x) f(x)$ is true, where $\sigma(u,x)$ is substituting a value from the domain $u \in \mathcal{U}$ is at least 1.5 times more than when it is false.  We can assume $\mathcal{U}$ is finite too.  Use the model in the previous problem.
\newline Let U = {1,2,3}, p = {1,3}, m = {(1,1),(2,1),(3,2)}
\begin{enumerate}
\item Determine $\models M x\ p(x)$
\item Determine $\models \forall \ x\ M y \ m(y,x) \rightarrow p(x)$
\end{enumerate}
\item Ursala, Kaiser, and Shilah are dogs.  We know the following:
\begin{enumerate}
\item Ursala is silver.
\item Shilah is gray and loves Kaiser.
\item Kaiser is either gray or silver (but not both) and loves Ursala.
\end{enumerate}
What does this sentence mean? $\exists x \exists y (gray(x) \wedge silver(y) \wedge loves(x,y)$.  Use resolution refutation to prove this.
\item Consider a robot that works in a mine -- it has to push some objects and not push others depending on a colored tag that is either green or red.  Here are the facts:
\begin{itemize}
\item If pushable objects are green, the non-pushable are red.
\item All objects are either green or red.
\item If there is a non-pushable object, the all pushable objects are green.
\item Object 1, a cart, is pushable.
\item Object 2, a pile of ore, is not pushable. 
\end{itemize}
Assume you're trying to prove that there is a red object. 
\begin{itemize}
\item Rewrite the statements into FOL (formal) and show their robotic equivalent (Python). \begin{itemize}
\item If pushable objects are green, the non-pushable are red. \newline
$(\exists x\ green(x) \wedge pushable(x)) \models (\exists y\ red(y) \wedge \neg pushable(y))$ \newline
$[\models, [\exists, x, [\wedge,[green,[x]],[pushable,[x]]]] [\exists, y, [\wedge,[red,[x]],[\neg, [pushable,[x]]]]$
\item All objects are either green or red. \newline
$\forall x \ green(x) \vee red(x)$
\item If there is a non-pushable object, the all pushable objects are green.
\item Object 1, a cart, is pushable.
\item Object 2, a pile of ore, is not pushable.
\end{itemize}
\item Convert to clausal form.
\item Use refutation to prove the there is a red object, by working {\it only} on the robotic equivalent.  Clearly in indicate the process.
\end{itemize}
\item Assume $\mathcal{U} = \{Alex, Bob, Cathy\}$, $M(x)$ means $x$ is a mechanic, $N(x)$ means $x$ works at NASA, $W(x,y)$ means $x$ worked with $y$, $I(x,y,z)$ means $x$ introduced $y$ to $z$.  Write constants $A,B,C$ to mean $Alex, Bob, Cathy$, respectively. Write the following in FOL:
\begin{itemize}
\item Cathy is a mechanic. Example:  $M(C)$
\item Bob is not a mechanic. \newline
$\neg M(B)$
\item Either Alex is a mechanic or Bob is, but I know Cathy works at NASA. \newline
$(M(A) \vee M(B)) \wedge M(C)$
\item Bob introduced Alex to Cathy, since Cathy works at NASA. \newline
$N(C) \models I(B,A,C)$
\item Someone is a mechanic, but everyone works at NASA.
$\exists x \forall y M(x) \wedge N(y)$
\item Bob introduced himself to Cathy. \newline
$I(B,B,C)$
\item Nobody has been introduced to Alex. \newline
$\forall x \forall y \neg I(x,y,A)$ 
\item If someone introduced Bob to Alex, then Bob isn't a mechanic. \newline
$\exists x I(x,B,A) \models \neg M(B)$
\item Nobody works with anyone here. \newline
$\forall x \forall y \neg W(x,y)$
\item Somebody works with Cathy, but it's not a mechanic, because Cathy works at NASA. \newline
$\exists x W(x,C) \wedge \neg M(x) \models N(C)$
\end{itemize}
\end{enumerate}
\end{document}
