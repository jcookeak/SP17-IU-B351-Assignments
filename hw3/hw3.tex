% LaTeX Article Template - customizing header and footer
\documentclass{article}

\newtheorem{thm}{Theorem}

% Set left margin - The default is 1 inch, so the following 
% command sets a 1.25-inch left margin.
\setlength{\oddsidemargin}{0.25in}

% Set width of the text - What is left will be the right margin.
% In this case, right margin is 8.5in - 1.25in - 6in = 1.25in.
\setlength{\textwidth}{6in}

% Set top margin - The default is 1 inch, so the following 
% command sets a 0.75-inch top margin.
\setlength{\topmargin}{-0.25in}

% Set height of the header
\setlength{\headheight}{0.3in}

% Set vertical distance between the header and the text
\setlength{\headsep}{0.2in}

% Set height of the text
\setlength{\textheight}{9in}

% Set vertical distance between the text and the
% bottom of footer
\setlength{\footskip}{0.1in}

% Set the beginning of a LaTeX document
\usepackage{multirow}

\usepackage{tikz}

\usepackage{fullpage}
\usepackage{graphicx}
\usepackage{amsthm}
\usepackage{url}
\usepackage{amssymb}
\usepackage{algpseudocode}
\usepackage{multicol}
\graphicspath{%
    {converted_graphics/}% inserted by PCTeX
    {/}% inserted by PCTeX
}
%%%%%%%%%%%%%%%%%%%%%%%%%%%%%




\begin{document}\title{Homework $3$\\ Computer Science \\ B351 Spring 2017\\ Prof. M.M. Dalkilic}         % Enter your title between curly braces
\author{Jonathon Cooke-Akaiwa}        % Enter your name between curly braces
\date{\today}          % Enter your date or \today between curly braces
\maketitle


% Redefine "plain" pagestyle
\makeatother     % `@' is restored as a "non-letter" character




% Set to use the "plain" pagestyle
\pagestyle{plain}
All the work herein is mine.

\section*{Homework Questions}
\begin{enumerate}
\item For a machine, working from both the goal and the start is not always feasible.  In some cases the goal is unknown.  A common strategy for a game is to utilize a heuristic.  A heuristic will evaluate a state and determine a score.  This score can be used as a best guess for how far away the goal is.  Through this and iteratively evaluating states until the goal is reached, a machine can solve these problems without knowing the actual goal.   
\item Minimax can be modified to acommodate multiple players by calculating a minimax score between all two player combinations, players A and B, A and C, and finally B and C.   By calculating the minimax of each two player combination these score can then be merged for the same state and an average score derived.  This average score could then be used like a regular minimax score.
\item See gobblet.py.
\end{enumerate}
\end{document}
